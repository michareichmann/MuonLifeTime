\begin{abstract}
 
In this experiment the life time of muons from the cosmic rays is measured. The cosmic rays interact with the atmosphere of the Earth where the muons are created. Due to their very low interaction rate they can reach the surface of the Earth.\par
The muon is stopped in a large block of scintillating material and life time is then measured as the time difference between the muon entering the system and the time the decaying electron or positron is detected. The resulting analogue signals are digitised with a desktop digitiser and recorded with a computer. \par
The goal of the experiment is set up a trigger logic to reduce the large number of background signals using elements from a NIM crate so the life time of the muon can be extracted from the data.
	
\end{abstract}





