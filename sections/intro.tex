\section{Introduction}
\subsection{Cosmic Ray}
Cosmic rays are a form of high-energy radiation, that mainly originates from outside
\wrapfig[r]{.3}{CosmicRayFlux}[Flux vs. energy of cosmic rays.][fig:ecr]
the Solar System. There is evidence that a significant fraction of primary cosmic rays originate from the supernova explosions of stars. \par 
Cosmic rays are composed of \SI{99}{\%} nuclei of atoms and \SI{1}{\%} solitary electrons.
\SI{90}{\%} of the nuclei are from hydrogen and \SI{9}{\%} from helium, i.e. protons and alpha particles respectively. The missing \SI{1}{\%} are nuclei from heavier atoms. A small fraction of cosmic rays are stable particles of antimatter such as positrons and antiprotons. The energy spectrum is shown in \ar{fig:ecr}. Ultra-high-energy cosmic rays can reach up to \SI{3e20}{\electronvolt} which is more then seven orders of magnitude more than the particles accelerated by the \ac{LHC}. This fact makes the Universe the strongest accelerators that exists and cosmic rays scientifically extremely interesting.

\subsection{Muon Production}
When primary cosmic rays reach the upper atmosphere of the Earth, they collide with atoms and molecules, mainly oxygen and nitrogen. Upon collision a cascade of particles, a so-called air shower is produced which is shown in \ar{fig:sho}. This secondary irradiation is mainly composed of x-rays, muons, protons, alpha particles, pions, electrons and neutrons, of which all stay within about one degree of the primary particle's path. The muons are mainly produced as decay products of the pions, which decay within in short distances of the order of meters. The resulting muons have velocities near the speed of light and due to their unusual low interaction rate with matter they can reach the Earth's surface.
\fig{.4}{AtmosphericCollision}[Primary cosmic particle collides with a molecule of atmosphere.][fig:sho]

	






